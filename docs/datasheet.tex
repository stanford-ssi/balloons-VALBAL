\documentclass[12pt, twocolumn]{article}

\usepackage[T1]{fontenc}
\usepackage{lmodern}
\usepackage{microtype}
%\usepackage[notmath]{sansmathfonts}
\usepackage{garamondx}
\usepackage{xcolor}
\usepackage{tikz}
%\usepackage{classico}
\usepackage{gensymb} 
\usepackage{amsmath}
\usepackage{amssymb}
\usepackage{enumitem}
%\usepackage{FiraSans}
\usepackage[english]{babel}
\usepackage[pangram]{blindtext}
\usepackage{wrapfig}


\usepackage[top=2.5in,left=0.62in,right=0.62in,bottom=0.62in]{geometry}


\makeatletter
\def\parsecomma#1,#2\endparsecomma{\def\page@x{#1}\def\page@y{#2}}
\tikzdeclarecoordinatesystem{page}{
    \parsecomma#1\endparsecomma
    \pgfpointanchor{current page}{north east}
    % Save the upper right corner
    \pgf@xc=\pgf@x%
    \pgf@yc=\pgf@y%
    % save the lower left corner
    \pgfpointanchor{current page}{south west}
    \pgf@xb=\pgf@x%
    \pgf@yb=\pgf@y%
    % Transform to the correct placement
    \pgfmathparse{(\pgf@xc-\pgf@xb)/2.*\page@x+(\pgf@xc+\pgf@xb)/2.}
    \expandafter\pgf@x\expandafter=\pgfmathresult pt
    \pgfmathparse{(\pgf@yc-\pgf@yb)/2.*\page@y+(\pgf@yc+\pgf@yb)/2.}
    \expandafter\pgf@y\expandafter=\pgfmathresult pt
}
\makeatother

\newcommand{\specialcell}[2][c]{%
  \begin{tabular}[#1]{@{}l@{}}#2\end{tabular}}
\newcommand{\specialcellr}[2][r]{%
  \begin{tabular}[#1]{@{}r@{}}#2\end{tabular}}

\usepackage{titlesec}

\titlespacing*{\subsection}
{0pt}{0ex plus 0.2ex minus .2ex}{0.5ex plus .2ex}

\usepackage{booktabs}
\usepackage{lipsum}
\def\labelitemi{--}

\begin{document}
\thispagestyle{empty}\sffamily

\noindent ValBal is a high altitude latex balloon platform that controls its altitude by venting lifting gas and dropping ballast mass. This extends the life of a low-cost latex balloon from a few hours to a record-breaking 5 days. It's designed to facilitate a broad range of high altitude research thanks to its unique ability to maintain and dynamically transition between altitudes all the way from ground level to it flight ceiling.

\begin{tikzpicture}[remember picture,overlay]
\fill[fill={rgb:red,190;green,30;blue,45}] (page cs:-1,1) rectangle (page cs:1,0.6);
\node (myfirstpic) at (page cs:-0.7,0.8) {\includegraphics[height=0.2\textheight]{ssilogo.png}};
\node[color=white,anchor=west] (datasheet) at (page cs:-0.45,0.913) {\sffamily\fontsize{16}{20}\selectfont \textsc{Stanford Student Space Initiative}};
\node[color=white,anchor=west] (datasheet) at (page cs:-0.45,0.76) {\sffamily\fontsize{35}{40}\selectfont ValBal};
\node[color=white,anchor=west] (datasheet) at (page cs:-0.45,0.683) {\sffamily\fontsize{16}{20}\selectfont Altitude controlled latex high altitude balloon system};
\end{tikzpicture}

%\textbf{ValBal} is a high-altitude balloon platform that 

\subsection*{\sffamily Highlights}
\vspace{1ex}
\def\power#1#2#3{\emph{Above $-$#1 \textdegree C:} #2~#3}
\begin{center}
\begin{tabular}{rl}
\toprule
\textbf{Altitude range} & \specialcell[t]{\emph{Standard:} 12.25--17 km\\\emph{Extended:} 0--23 km}\\\midrule
%\textbf{Peak power} & \specialcell[t]{\power{10}{3}{W}\\\power{40}{500}{mW}\\\power{60}{100}{mW}\\\power{70}{---}{}}\\\midrule
%\specialcellr[t]{\textbf{Cumulative}\\\textbf{energy}} & \specialcell[t]{\emph{24~hours:} 68~Wh\\\emph{48~hours:} 56~Wh\\\emph{72~hours:} 44~Wh}\\\midrule
\textbf{Endurance} & \specialcell[t]{\emph{No payload:} 5~days\\\emph{1~kg payload:} 3~days\\\emph{5~kg payload:} 1~day}\\\midrule
\textbf{Communications} & \specialcell[t]{  \begin{minipage}[t]{0.5\linewidth}
    \begin{itemize}[topsep=0pt,itemsep=-1ex,partopsep=1ex,parsep=1ex,leftmargin=2ex]
    \item Iridium, 8\textcent/50 byte.
    \item Custom 433~MHz radio, ~5~kBps line-of-sight.
    \end{itemize}\end{minipage}\vspace{1mm} }\\\midrule
    \multicolumn{2}{c}{\footnotesize\begin{minipage}[t]{0.9\linewidth}\textbf{\small Notes:}
\begin{itemize}[parsep=1ex,itemsep=-1ex]
\item Many of the design parameters are flexible, and can be customized on a per mission basis.
%\item In particular, energy can be easily increased at a density of 300~Wh/kg, at the cost of payload; similarly for peak power.%; peak power at 
\item An endurance of five days was demonstrated with zero payload. This reflects our current best mission, but we believe that longer missions are feasible with the system.
\end{itemize}\end{minipage}\vspace{1mm}}\\\bottomrule
\end{tabular}
\end{center}
{\footnotesize 

\paragraph{\sffamily Ease of assembly:} Valbal's incredible lorem ipsum makes it \Blindtext[1][5]

\paragraph{\sffamily Some stuff:} Some stuff indeed \Blindtext[1][3]

%\section*{\sffamily Highlights}
\begin{center}\includegraphics[width=0.5\linewidth]{render.jpg}\end{center}
\begin{center}\includegraphics[width=\linewidth,trim={1cm 0.3cm 2cm 1.5cm},clip]{trajfig.png}\end{center}

\paragraph{\sffamily Control capabilities:} The above plot demonstraints an examply of what a 5-day flight out of california can look like through monte-carlo simlation with NOAA wind data. The red lines show the result of a basic flight profile with no objective, while the blue shows a flight plan optimized for longitudinal distance.
}


\end{document}